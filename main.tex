\documentclass[12pt]{article}
\usepackage[utf8]{inputenc}

% --- Page Requirements ---
\usepackage[margin=1in]{geometry}
\usepackage{indentfirst}
\usepackage{setspace}

% --- Sources ---
\usepackage[backend=biber,style=ieee]{biblatex}
\addbibresource{sources.bib}

% --- Title Page ---
\usepackage{titling}
\newcommand{\subtitle}[1]{%
	\posttitle{%
		\par\end{center}
	\begin{center}\large#1\end{center}
	\vskip0.1em}}%

\title{Individual Paper Assignment: \\
    International Environmental Regulation in the Mining Sector}
\subtitle{PEGN 430A}
\author{Tyler Singleton}
\date{07 April 2022}

\begin{document}

% --- Title Page --- %
\maketitle
\newpage

% --- Body --- %

% Requirements
% 6-8 Pages
% Double Spaced
% 3 UN Sustainable Development Goals
% 8 Sources - 5 Scholarly
% 12pt Font
% 1in Margins

\doublespacing

\renewcommand{\abstractname}{Introduction}
\begin{abstract}
% Current or Developing topic and why I selected it
\normalsize
\noindent
As the world is progressing toward environmentally sustainable growth, there is a significant push concerning reduced and carbon neutral development \cite{Report:IEA_Executive_Summary, News:Environmental_Regulation}. Electrification and cleaner energy technology and systems are expected to increase the demand for mineral resources by sixfold in the next two decades according to a recent executive summary (2021) from IEA \cite{Report:IEA_Executive_Summary}. In short the total demand for copper and rare earth elements will increase by 40\%, nickel and cobalt by 60-70\%, and 90\% for lithium \cite{Report:IEA_Executive_Summary}. Forming the basis of this paper, the extraction of these critical elements are fundamental to the United Nations Sustainability Development Goals (UNSDG) and contribution to the Paris Agreement as electrification is vital in achieving carbon neutrality. Published on the United Nations Environment Programme (UNEP) website, an opinion piece from the UNEP Economy Division Director \cite{News:Environmental_Regulation}, highlights the lack of comprehensive environmental standards within the mining sector as compared to human rights. She denotes a need for government, industry leaders, and society to promote a conscious effort to strengthen environmental standards in the mining industry that proliferate a comprehensive and sustainable green transition \cite{News:Environmental_Regulation}. From this, this paper will focus on international governance of environmental regulations within the mining sector. 
\end{abstract}

\section{Background}
% Describe the background of the topic and applicable US and/or international law
In \citetitle{Article:Mining_Industry}, \citeauthor{Article:Mining_Industry} surmises the environmental impact of mining has been unrecognized throughout history -- it was in the last quarter of the twentieth century that we have begun a global push toward understanding the environmental footprint mining has \cite{Article:Mining_Industry}. Development of environmental legislation to combat the drainage of toxic substances and minerals from mining operations; and to provide environmental remediation, clean up, and rehabilitation began in 1980 when the United States congress established CERCLA (Comprehensive Environmental Response, Compensation, and Liability Act of 1980, Pub. L. No. 96-510) \cite{Article:Mining_Industry}. However, legislation supporting substantial environmental reform within the mining sector predominately fails to extend past North America and Western Europe \cite{Article:Mining_Industry}. While international frameworks, recommendations, and codes of conduct exist to proliferate environmentally sustainable practices; they lack enforcement and are mostly voluntary \cite{Book:Governance_Resources}.

The International Resource Panel (IRP) reported an urgent need to establish an effective global governance of the mining sector \cite{Book:MRG_Sustainable_Report}. In their report, they acknowledge the economies of developing countries that rely on the mining industry provide significant challenges as resource scarcity, volatile prices, social conflicts, and instability of local governance can easily undermine development outcomes \cite{Book:MRG_Sustainable_Report}. Additionally, initiatives to combat these challenges provide an ineffective piece-meal approach \cite{Book:MRG_Sustainable_Report}. Since the international demand for minerals have exponentially increased with global gross domestic product (GDP) \cite{Article:Mining_Industry, Book:Governance_Resources, Book:MRG_Sustainable_Report}, the implications for countries to seek lucrative mining operations over environmental concerns necessitates stronger regulation. 

With developing countries engaging a transition to electrification to meet the Paris Agreement for climate change, ore-bodies containing copper, nickel, cobalt, and lithium are becoming a crucial resource \cite{Book:Governance_Resources}. \cite{Report:IEA_Executive_Summary}. Notably, lower quality ore-bodies are used to extract these minerals as the higher quality ore-bodies have been used \cite{News:Environmental_Regulation, Report:IEA_Executive_Summary, Article:Mining_Industry, Book:Governance_Resources, Book:Metal_Recycling}. Although the extraction of ores have increased in complexity, the real price of minerals have been relatively stable thanks to breakthroughs in ore processing and mining technology which keep extraction up with demand \cite{Book:Governance_Resources}. However, for developing nations to receive the same benefits that developed countries have obtained through unbridled extraction, and with demand expected to quadruple \cite{Report:IEA_Executive_Summary}, it is unclear if continuous scientific development will continue to aid in meeting global demand \cite{Book:Governance_Resources}. This may result resource exhaustion where the cost to continue extraction is no longer feasibly profitable \cite{Book:Governance_Resources}. As a result of this, developing nations establishing extraction quotas in a similar manner to the Organization of the Petroleum Exporting Countries (OPEC) is a reasonable proposal given by \citeauthor{Book:Governance_Resources} \cite{Book:Governance_Resources}. Under the auspice of the United Nations, such a system allows developing nations to maintain their sovereignty and regulate the environmental footprint from the mining sector \cite{Book:Governance_Resources}. This is paramount to achieve UNSDGs \cite{Book:MRG_Sustainable_Report}.

\section{Discussion}
% Application of the pertinent sections of the law and at least three UNSDGs that apply
In June of 1992, the United Nations held a conference in Rio de Janerio. This conference spurred two important documents to lay the foundation of international law on environmental sustainability -- first is the U.N. Conference on Environment and Development, \textit{Rio Declaration on Environment and Development}, U.N. Doc A/CONF.151/26/Rev.1 (Vol. I), annex I (Aug. 12 1992) [hereinafter \textit{Rio Declaration}]; and U.N. Conference on Environment and Development, \textit{Agenda 21}, U.N. Doc A/CONF.151/26/Rev.1 (Vol. I), annex II (Aug. 12 1992). The Rio Declaration set twenty-seven defining principles to recognize an international partnership with the goal of protecting the global environmental integrity. Following up on this, Agenda 21 sets forth the underlying framework of how to build upon this partnership and address these challenges \cite{Website:United_Nations_Conference}, and signed by 179 countries \cite{Website:United_Nations_Conference}. Although this document is non-binding, it provides the legal framework to sustainable development goals that have penetrated 112 multilateral treaties and over 300 international conventions \cite{Article:Development_International_Law}. However, most mentions of sustainable development mostly reside in the preamble or language is too vague to effectively enforce \cite{Article:Development_International_Law}, but this does support a case for customary international law \cite{Article:Development_International_Law}.

With this, the United Nations have reported that a establishing an international and enforceable governance of the mining sector to promote sustainable development will achieve multiple UNSDGs \cite{Book:MRG_Sustainable_Report}. In particular the IRP describes the following goals can be met: Goal 1 -- No Poverty -- from investing the revenue generated through mining toward the local economy and social development; Goal 8 -- Decent Work and Economic Growth -- by providing opportunities of employment to local communities; Goal 9 -- Industry, Innovation, and Infrastructure -- from technological advancements in construction of infrastructure to support a growing economy; and Goals 6, 7, 13, 15, and 16 \cite{Book:MRG_Sustainable_Report} as these goals are all highly intertwined into the mining sector.

In \citetitle{Book:Governance_Resources}, \citeauthor{Book:Governance_Resources} proposes a cap and trade solution to achieve this international governance. After reviewing eleven different instruments to enforce, \citeauthor{Book:Governance_Resources} reasons a cap and trade is the most effective solution as it maintains a nations sovereignty \cite{Book:Governance_Resources}. This would work much in the same framework as OPEC, where developing nations would negotiate a mineral extraction quota amongst themselves that would allow for price stability, better documentation of the environmental impacts (current documentation surrounding mining operations are scattered among private and public collections, and usually remain undisclosed \cite{Book:MRG_Sustainable_Report}), and an agreement between countries with corresponding interests will likely be easier to generate than a document which accounts for a wider range of nations \cite{Book:Governance_Resources}. By doing so, and with the recognition of the United Nations, a UNSDGs can be achieved. 

One of the key issues with international mining regulations is the lack of recording standards among developing nations \cite{Article:Mining_Industry, Book:Mining_Records}. In Europe, there are 63 indicators criteria indicators (C&I) to capture environmental issues related to mining, the United Nations propose a set of 50 C&I, the Sustainability Reporting Guidelines recommends 150 C&I, and the World Bank recommends 10 C&I \cite{Book:Mining_Records}. To combat these issues, the Mining Minerals and Sustainable Development (MMSD) project was created to examine the records across all mining operations and into a comprehensive list to compare \cite{Book:Mining_Records}. Part of this operation spurred the International Council on Mining and Minerals (ICMM) which uses developed a partnership across industry leaders within the mining sector to develop and standardize reporting and promote voluntary practices of sustainable mining operations \cite{Book:Mining_Records}. 

% --- A Case Study --- %


\section{Conclusion}
% Synthesis and recommendations for the future

Under the Paris Agreement to the United Nations Framework Convention on Climate Change, Dec. 12, 2015, T.I.A.S. No. 16-1104, an international agreement to adopt a push to reduce GHG emissions has spurred developed countries toward electrification. This has considerable environmental impacts on developing countries as they have the most to borne the from repercussions of climate change \cite{Report:IEA_Executive_Summary}. Developed countries striving for electrification will quadruple the demand of minerals such as copper, nickel, boron, and lithium \cite{Report:IEA_Executive_Summary, News:Environmental_Regulation} to improve their infrastructure and development of electric batteries and motors \cite{Report:IEA_Executive_Summary}. However, the quality of ore-bodies capable to support this large increase of demand are vastly diminishing, and leading to resource scarcity \cite{Book:Governance_Resources, Book:Metal_Recycling, Report:IEA_Executive_Summary, News:Environmental_Regulation}. Mining these minerals requires approximately four times the amount of rock needs to be removed \cite{Article:Mining_Industry}, and along every step of the extraction process, environmental impacts are realized \cite{Article:Mining_Industry}. Although scientific breakthroughs have allowed real prices of ore to remain relatively stable, it is still unclear if this will remain so to meet future demand \cite{Book:Governance_Resources}.

With these facts, it is important to develop a new legal framework for international governance of mineral extraction. This framework has the potential to be established and enforceable via customary law though the extensive discussion of sustainable development across a multitude of international treaties and conferences \cite{Article:Development_International_Law}. However, the legal framework should reside within the United Nations, but respective of developing countries sovereignty. It is proposed that a cap and trade system where developing nations can establish a quota amongst themselves would be the most effective means of international governance \cite{Book:Governance_Resources}. 

Developing a successful legal framework to promote environmentally conscious mining will achieve numerous UNSDGs as they are closely tied to the mining sector \cite{Book:MRG_Sustainable_Report}. Stable pricing of minerals assists in stabilizing governments within developing nations which allows for the revenue streams from mining to be invested into the local economy for infrastructure, social development, job opportunities, education, and poverty \cite{Book:Governance_Resources}. Currently, most treaties establishing sustainable development are only voluntary and unenforceable due to vague language within the law \cite{Article:Development_International_Law}. Future treaties need to be more concise with the metrics for sustainable development, and a holistic approach needs to established. Part of this push for a holistic approach requires industry leaders to actively pursue and utilize environmental conscious operations. The ICMM is a council that allows for this purpose and by have a partnership of industry leaders that account of over a third of the global industry \cite{Website:ICMM}. 

% --- Bibliography ---
\newpage
\printbibliography


\end{document}
